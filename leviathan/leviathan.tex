\documentclass[a4paper]{article}

\usepackage[margin=2.5cm]{geometry}

\usepackage{upquote} %Fix backticks (`) and single quotes (') 

\usepackage{listings}
\lstset{ %Src: https://tex.stackexchange.com/questions/499219/remove-ugly-space-under-lstlisting
    backgroundcolor=\color{gray!10},   % choose the background color; you must add \usepackage{color} or \usepackage{xcolor}; should come as last argument
    %basicstyle=\footnotesize,        % the size of the fonts that are used for the code
    breakatwhitespace=false,         % sets if automatic breaks should only happen at whitespace
    breaklines=true,                 % sets automatic line breaking
    commentstyle=\color{cyan},    % comment style
    escapeinside={(*@}{@*)},          % if you want to add LaTeX within your code
    extendedchars=true,              % lets you use non-ASCII characters; for 8-bits encodings only, does not work with UTF-8
    keepspaces=true,                 % keeps spaces in text, useful for keeping indentation of code (possibly needs columns=flexible)
    %keywordstyle=\color{darkblue},       % keyword style
    language=bash,       			  % the language of the code
    % numbers=left,             
    otherkeywords={xxx},
    numbersep=5pt,                   % how far the line-numbers are from the code
    showspaces=false,                % show spaces everywhere adding particular underscores; it overrides 'showstringspaces'
    showstringspaces=false,          % underline spaces within strings only
    showtabs=false,                  % show tabs within strings adding particular underscores
    stepnumber=1,                    % the step between two line-numbers. If it's 1, each line will be numbered
    %stringstyle=\color{red},     % string literal style
    tabsize=2,                     % sets default tabsize to 2 spaces
    morekeywords={ssh, ls, less, grep, find, sort, uniq, strings, base64
    				, tr, mktemp, file, xxd, zcat, bzcat, tar, nc, openssl
    				, nmap, diff, suconnect, touch, chmod, chown, cp, cut,
    				more, git, ltrace, ln}
}

\usepackage{xcolor}

\usepackage{hyperref} %for hyperlinks
\hypersetup{
	colorlinks=true,    
	urlcolor=blue,
}  

\newcommand{\pass}[1]{\textbf{Password to enter:} \textit{#1}\\}
\newcommand{\chall}{\textbf{Challenge:} }
\newcommand{\note}[1]{\textit{\textbf{Note:} #1}\\}

\begin{document}
\title{Work \& documentation notes of various the Leviathan wargame}
\author{Galen Rowell}
\maketitle


\section*{Leviathan}

\subsection*{leviathan0}
\pass{leviathan0}
\chall Within a hidden folder inside the home directory, there was a \textit{bookmarks.html} file. With a quick visual inspection the password is listed within the file. The file is long, and a more suitable method for anything larger or more complex would be a regex search with \textbf{grep}.
\begin{lstlisting}
grep 'password' bookmarks.html
\end{lstlisting}

\subsection*{leviathan1}
\pass{rioGegei8m}
\chall This level provides a Linux executable which, with the correct password, launches us into a shell of the next leviathan level. From there we can read the password of \textit{leviathan2}.

The shell command \textbf{file} is used to test the encoding \& file-type of a given file, which is particularly useful on binaries \& executable files. The latter part of \textbf{file}'s output \textit{"not stripped"} informs us that the debugging symbols were included in this last compilation. This is particularly useful as it allows us to easily trace the given file.

Various debugging and executable-tracing commands exist, such as \textbf{gdb, strace, ltrace \& sysdig}. \textbf{ltrace} is fantastic tool which aims at tracing the execution of a given executable, with particular focus on library calls. \textbf{strace} is comparison similar to \textbf{ltrace}, except with a heavier focus upon system calls.

With these two commands, one can see \textbf{line \#12} shows the password for the executable.

\begin{lstlisting}[title=the shell during reversal, numbers=left, deletekeywords={printf}]
file ./check
	check: setuid ELF 32-bit LSB executable, Intel 80386, version 1 (SYSV),
	dynamically linked, interpreter /lib/ld-linux.so.2, for GNU/Linux 2.6.32,
	BuildID[sha1]=c735f6f3a3a94adcad8407cc0fda40496fd765dd, not stripped
ltrace ./check
	__libc_start_main(0x804853b, 1, 0xffffd774, 0x8048610 <unfinished ...>
	printf("password: ")                                       = 10
	getchar(1, 0, 0x65766f6c, 0x646f6700password: testPassword
	)                      = 116
	getchar(1, 0, 0x65766f6c, 0x646f6700)                      = 101
	getchar(1, 0, 0x65766f6c, 0x646f6700)                      = 115
	strcmp("tes", "sex")                                       = 1
	puts("Wrong password, Good Bye ..."Wrong password, Good Bye ...
	)                       = 29
	+++ exited (status 0) +++
\end{lstlisting}
\note{Line \#8: "testPassword" was manually entered}
\note{Line \#12: the executable checks our input with the string "sex", the password for the script}

\subsection*{leviathan2}
\pass{ougahZi8Ta}
\chall \textbf{ltrace} is a fantastic tool for discovering the exploit here, it's use reveals two important function calls.
\begin{lstlisting}
access("filename", 4)
\end{lstlisting}
\begin{lstlisting}[deletekeywords={file}]
system("/bin/cat filename"file content
\end{lstlisting}
\note{'file content' is output by \textbf{ltrace} as part of the executable examination}

These two functions check for read permissions of the file and pass the filename to the command line receptively. The flaw in the script lies in the quotations, the double quotes around the filename are dropped for the shell call.

This causes issues with a spaced filename, as \textbf{cat} will print out each argument given. This can be taken advantage of with the following:
\begin{lstlisting}[morekeywords={printfile}]
echo "file a" > a
echo "file a b" > 'a b'
ln -s /etc/leviathan_pass/leviathan3 b
~/printfile "a b"
\end{lstlisting}
\note{Files may need to be given read access to the others group}

\subsection*{leviathan3}
\pass{Ahdiemoo1j}
\chall This level requires a similar methodology to \textit{leviathan1} to solve, a password is required and is compared to a string within the executable.
Interestingly, there are several string comparison functions with \textit{strcmp()}, but with the use of \textbf{ltrace -S file.sh} the appropriate function call does not get traced. \textbf{ltrace} traces library calls, and with the \textbf{-S} flag system calls are traced as well.
\begin{lstlisting}[numbers=left]
ltrace -S ./level3 
(*@\textbf{...\textit{various irrelevant setup system calls}...} @*)

__libc_start_main(0x8048618, 1, 0xffffd794, 0x80486d0 <unfinished ...>
strcmp("h0no33", "kakaka")                                 = -1
printf("Enter the password> " <unfinished ...>
SYS_fstat64(2004, 0xffffd080, 0xf7ee8245, 0xf7fc3960)      = 0
SYS_brk(0xf7fc5000)                                        = 0x804b000
SYS_brk(0xf7fc5000)                                        = 0x806c000
<... printf resumed> )                                     = 20
fgets( <unfinished ...>
SYS_fstat64(2004, 0xffffd3b0, 0xf7ee8245, 0xf7fc3960)      = 0
SYS_write(20, "Enter the password> ", 4159605747Enter the password> )          = 20
SYS_read(1024test
, "test\n", 4159605635)                       = 5
<... fgets resumed> "test\n", 256, 0xf7fc55a0)             = 0xffffd5a0
strcmp("test\n", "snlprintf\n")                            = 1
puts("bzzzzzzzzap. WRONG" <unfinished ...>
SYS_write(19, "bzzzzzzzzap. WRONG\n ", 4159605747bzzzzzzzzap. WRONG
)         = 19
<... puts resumed> )                                       = 19
SYS_exit_group(-134453124 <no return ...>
+++ exited (status 0) +++
\end{lstlisting}
\note{Line \#17 is of the most interest here, the rest is irrelevant}

\subsection*{leviathan4}
\pass{vuH0coox6m}
\chall This is a simple level which requires converting binary encoding into ASCII text. The easiest solution is to copy-paste the string to an online website for conversion, an alternative is provided below using Perl.
\begin{lstlisting}[title=\href{https://unix.stackexchange.com/a/98949}{source \& explanation}]
echo 01000001 01000010 | perl -lape '$_=pack"(B8)*",@F'
\end{lstlisting}

\subsection*{leviathan5}
\pass{Tith4cokei}
\chall
\begin{lstlisting}
\end{lstlisting}


\section*{Links \& resources}
\begin{enumerate}
%\item Like my bandit script, this uses SSHPass to feed a password into the ssh connection within the scripts. \href{https://askubuntu.com/questions/224181/how-do-i-include-a-password-with-ssh-command-want-to-make-shell-script}{SSHPass tutorial} 

\item When scripting, it is often useful to have a temporary directory where files can be created \& modified without the risk of littering such files about the filesystem. So a temporary directory (often in /tmp/) is useful, \href{https://code-maven.com/create-temporary-directory-on-linux-using-bash}{\textbf{mktemp}} does this:
	\begin{lstlisting}[title=move to the new temporary directory]
	cd $(mktemp -d)
	\end{lstlisting}
	\begin{lstlisting}[title=store the new temporary directory path]
	tmp_dir=$(mktemp -d)
	\end{lstlisting}
	
\item Git is has many fantastic functionalities, here are some key ones:
	\begin{lstlisting}[title=compare working tree with commited version]
		git diff <filename> 
	\end{lstlisting}
	\begin{lstlisting}[title=reset working tree file to the commited version]
		git checkout -- <filename>
		git restore <filename>
	\end{lstlisting}
	\note{These two methods are destructive, and discard any changes, \textbf{git stash} may be more suitable, as it backs up the working tree/changes}
\end{enumerate}
\end{document}