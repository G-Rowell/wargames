\documentclass[a4paper]{article}

\usepackage[margin=2.5cm]{geometry}

\usepackage{upquote} %Fix backticks (`) and single quotes (') 

\usepackage{listings}
\lstset{ %Src: https://tex.stackexchange.com/questions/499219/remove-ugly-space-under-lstlisting
    backgroundcolor=\color{gray!10},   % choose the background color; you must add \usepackage{color} or \usepackage{xcolor}; should come as last argument
    %basicstyle=\footnotesize,        % the size of the fonts that are used for the code
    breakatwhitespace=false,         % sets if automatic breaks should only happen at whitespace
    breaklines=true,                 % sets automatic line breaking
    commentstyle=\color{cyan},    % comment style
    escapeinside={(*@}{@*)},          % if you want to add LaTeX within your code
    extendedchars=true,              % lets you use non-ASCII characters; for 8-bits encodings only, does not work with UTF-8
    keepspaces=true,                 % keeps spaces in text, useful for keeping indentation of code (possibly needs columns=flexible)
    %keywordstyle=\color{darkblue},       % keyword style
    language=bash,       			  % the language of the code
    % numbers=left,             
    otherkeywords={xxx},
    numbersep=5pt,                   % how far the line-numbers are from the code
    showspaces=false,                % show spaces everywhere adding particular underscores; it overrides 'showstringspaces'
    showstringspaces=false,          % underline spaces within strings only
    showtabs=false,                  % show tabs within strings adding particular underscores
    stepnumber=1,                    % the step between two line-numbers. If it's 1, each line will be numbered
    %stringstyle=\color{red},     % string literal style
    tabsize=2,                     % sets default tabsize to 2 spaces
    morekeywords={ssh, ls, less, grep, find, sort, uniq, strings, base64
    				, tr, mktemp, file, xxd, zcat, bzcat, tar, nc, openssl
    				, nmap, diff, suconnect, touch, chmod, chown, cp, cut,
    				more, git, ltrace, ln}
}

\usepackage{xcolor}

\usepackage{hyperref} %for hyperlinks
\hypersetup{
	colorlinks=true,    
	urlcolor=blue,
}  

\newcommand{\pass}[1]{\textbf{Password to enter:} \textit{#1}\\}
\newcommand{\chall}{\textbf{Challenge:} }
\newcommand{\note}[1]{\textit{\textbf{Note:} #1}\\}

\begin{document}
\title{Work \& documentation notes of various the Krypton wargame}
\author{Galen Rowell}
\maketitle


\section*{Krypton}

\subsection*{krypton1}
\pass{KRYPTONISGREAT}
\chall Within .
\begin{lstlisting}
grep 'password' bookmarks.html
\end{lstlisting}

\section*{Links \& resources}
\begin{enumerate}
%\item Like my bandit script, this uses SSHPass to feed a password into the ssh connection within the scripts. \href{https://askubuntu.com/questions/224181/how-do-i-include-a-password-with-ssh-command-want-to-make-shell-script}{SSHPass tutorial} 

\item When scripting, it is often useful to have a temporary directory where files can be created \& modified without the risk of littering such files about the filesystem. So a temporary directory (often in /tmp/) is useful, \href{https://code-maven.com/create-temporary-directory-on-linux-using-bash}{\textbf{mktemp}} does this:
	\begin{lstlisting}[title=move to the new temporary directory]
	cd $(mktemp -d)
	\end{lstlisting}
	\begin{lstlisting}[title=store the new temporary directory path]
	tmp_dir=$(mktemp -d)
	\end{lstlisting}
	
\item Git is has many fantastic functionalities, here are some key ones:
	\begin{lstlisting}[title=compare working tree with commited version]
		git diff <filename> 
	\end{lstlisting}
	\begin{lstlisting}[title=reset working tree file to the commited version]
		git checkout -- <filename>
		git restore <filename>
	\end{lstlisting}
	\note{These two methods are destructive, and discard any changes, \textbf{git stash} may be more suitable, as it backs up the working tree/changes}
\end{enumerate}
\end{document}