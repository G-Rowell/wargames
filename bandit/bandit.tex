\documentclass[a4paper]{article}

\usepackage{listings}

\newcommand{\pass}[1]{\textbf{Password to enter:} \textit{#1}\\}
\newcommand{\chall}{\textbf{Challenge:} }

\begin{document}
\title{Work \& documentation notes of various wargames}
\author{Galen Rowell}
\maketitle


\section{Bandit}
\subsection{Levels}

\subsubsection{bandit0}
\pass{bandit0}
\chall Solved using the SSH command, which included use of flags to set user \& port.
\begin{lstlisting}[language=Bash]
ssh bandit0@bandit.labs.overthewire.org -p 2220
\end{lstlisting}

\subsubsection{bandit1}
\pass{boJ9jbbUNNfktd78OOpsqOltutMc3MY1}
\chall Reading a file named '-', this was problematic due to many common shell commands using '-' to prefix an option or flag.
\begin{lstlisting}[language=Bash]
cat ./-
\end{lstlisting}

\subsubsection{bandit2}
\pass{CV1DtqXWVFXTvM2F0k09SHz0YwRINYA9}
\chall there are spaces in the password filename

\subsubsection{bandit3}
\pass{UmHadQclWmgdLOKQ3YNgjWxGoRMb5luK}
\chall password file is a 'dotfile' (prepended by '.' and hidden) and in a sub-directory of home

\subsubsection{bandit4}
\pass{pIwrPrtPN36QITSp3EQaw936yaFoFgAB}
\chall Password is hidden in one of '/home/bandit4/inhere/-file{0,9}'. They contain special shell characters that interfere with the terminal environment. The use of 'less' aids, as it prompts before reading a binary file and provides a somewhat isolated viewing environment.

\subsubsection{bandit5}
\pass{koReBOKuIDDepwhWk7jZC0RTdopnAYKh}
\chall The password is in 

\subsection{Links \& resources}
\begin{enumerate}
\item SSHPass: https://askubuntu.com/questions/224181/how-do-i-include-a-password-with-ssh-command-want-to-make-shell-script
\end{enumerate}
\end{document}