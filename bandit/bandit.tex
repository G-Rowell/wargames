\documentclass[a4paper]{article}

\usepackage{listings}
\lstset{ %Src: https://tex.stackexchange.com/questions/499219/remove-ugly-space-under-lstlisting
    backgroundcolor=\color{gray!10},   % choose the background color; you must add \usepackage{color} or \usepackage{xcolor}; should come as last argument
    %basicstyle=\footnotesize,        % the size of the fonts that are used for the code
    breakatwhitespace=false,         % sets if automatic breaks should only happen at whitespace
    breaklines=true,                 % sets automatic line breaking
    commentstyle=\color{cyan},    % comment style
    escapeinside={\%*}{*)},          % if you want to add LaTeX within your code
    extendedchars=true,              % lets you use non-ASCII characters; for 8-bits encodings only, does not work with UTF-8
    keepspaces=true,                 % keeps spaces in text, useful for keeping indentation of code (possibly needs columns=flexible)
    %keywordstyle=\color{darkblue},       % keyword style
    language=bash,       			  % the language of the code
    % numbers=left,             
    otherkeywords={xxx},
    numbersep=5pt,                   % how far the line-numbers are from the code
    showspaces=false,                % show spaces everywhere adding particular underscores; it overrides 'showstringspaces'
    showstringspaces=false,          % underline spaces within strings only
    showtabs=false,                  % show tabs within strings adding particular underscores
    stepnumber=1,                    % the step between two line-numbers. If it's 1, each line will be numbered
    %stringstyle=\color{red},     % string literal style
    tabsize=2,                     % sets default tabsize to 2 spaces
    morekeywords={ssh, ls, less, grep, find, sort, uniq, strings, base64}
}

\newcommand{\pass}[1]{\textbf{Password to enter:} \textit{#1}\\}
\newcommand{\chall}{\textbf{Challenge:} }

\usepackage{xcolor}

\begin{document}
\title{Work \& documentation notes of various wargames}
\author{Galen Rowell}
\maketitle


\section{Bandit}
\subsection{Levels}

\subsubsection{bandit0}
\pass{bandit0}
\chall Solved using the \textbf{ssh} command, which included use of flags to set user \& port.
\begin{lstlisting}
ssh bandit0@bandit.labs.overthewire.org -p 2220
\end{lstlisting}

\subsubsection{bandit1}
\pass{boJ9jbbUNNfktd78OOpsqOltutMc3MY1}
\chall Reading a file named '-', this was problematic due to many common shell commands using '-' to prefix an option or flag.
\begin{lstlisting}
cat ./-
\end{lstlisting}

\subsubsection{bandit2}
\pass{CV1DtqXWVFXTvM2F0k09SHz0YwRINYA9}
\chall With spaces in a filename, shell programs will interpret the input as several arguments (instead of one space-delimited string). This issue can be solved two ways.
\begin{lstlisting}
cat 'spaced filename'
\end{lstlisting}
\begin{lstlisting}
cat spaced\ filename
\end{lstlisting}

\subsubsection{bandit3}
\pass{UmHadQclWmgdLOKQ3YNgjWxGoRMb5luK}
\chall The file is prepended by a '.', which causes it to be hidden from most views. The \textit{-A} flag for \textbf{ls} will show all hidden files except '.' \& '..', which are part of the directory itself.
\begin{lstlisting}
ls -A1
\end{lstlisting}

\subsubsection{bandit4}
\pass{pIwrPrtPN36QITSp3EQaw936yaFoFgAB}
\chall The file is hidden in one of '~/inhere/-file{0,9}'. They contain special characters that interfere with the terminal environment. The use of \textbf{less} aids, as it prompts before reading a binary file and provides somewhat of a sandbox to prevent the tty from being broken.
\begin{lstlisting}
less ./-file0[0-9]
\end{lstlisting}
\textit{Note: use :n when inside \textbf{less} to go the next file}

\subsubsection{bandit5}
\pass{koReBOKuIDDepwhWk7jZC0RTdopnAYKh}
\chall The file is within one of many sub-folders, with human readable encoding and a file size of '1033' bytes. The use of \textbf{ls} with the recursive flag \textit{-R}, combined with \textbf{grep} to select the file with the given size solves this problem.
\begin{lstlisting}
ls -Al -R | grep --color -C 5 -e '1033'
\end{lstlisting}

\subsubsection{bandit6}
\pass{DXjZPULLxYr17uwoI01bNLQbtFemEgo7}
\chall The file is somewhere on the server, so we should search recursively from the root of the drive. We are given the owner name, group name and size of the file, which we can plug into \textbf{find} to find the file.
\begin{lstlisting}
find / -group bandit6 -size 33c 2>&1 | grep -v "Permission denied"
\end{lstlisting}
\textit{Note: The use of a terminal redirect and \textbf{grep} remove the output of excessive file permission warnings}

\subsubsection{bandit7}
\pass{HKBPTKQnIay4Fw76bEy8PVxKEDQRKTzs}
\chall This level is a simple grep search for the word \textit{millionth} in a large keyword text file.

\subsubsection{bandit8}
\pass{cvX2JJa4CFALtqS87jk27qwqGhBM9plV}
\chall The password is the only line that occurs once within an unordered text file.
\begin{lstlisting}
sort data.txt | uniq -u
\end{lstlisting}
\textit{Note: The -u flag of \textbf{uniq} ensures only lines of 1 occurrence are printed}

\subsubsection{bandit9}
\pass{UsvVyFSfZZWbi6wgC7dAFyFuR6jQQUhR}
\chall The given file is a binary encoded file, IE. it is not in plaintext or easy to read. \textbf{strings} will only print human-readable strings from a given input, and the use of \textbf{grep} will limit the output to a manageable size.
\begin{lstlisting}
strings data.txt | grep -Ee [=]+
\end{lstlisting}
\textit{Note: The [=]+ pattern of \textbf{grep} searches for one or more occurances of \textit{=} in each line, EG. \textit{=}, \textit{====} or \textit{========}}

\subsubsection{bandit10}
\pass{truKLdjsbJ5g7yyJ2X2R0o3a5HQJFuLk}
\chall The file is encoded in base 64, which can be encoded and decoded using the \textbf{base64} program.
\begin{lstlisting}
base64 -d data.txt
\end{lstlisting}

\subsubsection{bandit11}
\pass{}
\chall

\subsubsection{bandit12}
\pass{}
\chall

\subsubsection{bandit14}
\pass{}
\chall

\subsubsection{bandit14}
\pass{}
\chall

\subsection{Links \& resources}
\begin{enumerate}
\item SSHPass: https://askubuntu.com/questions/224181/how-do-i-include-a-password-with-ssh-command-want-to-make-shell-script
\end{enumerate}
\end{document}